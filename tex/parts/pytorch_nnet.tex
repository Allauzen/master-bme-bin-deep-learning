\begin{frame}{pytorch in three concepts}
  \begin{block}{A \textit{Tensor} is a tensor}
    \begin{itemize}
    \item Similar to numpy’s \textit{ndarrays}, but can be used on a
      GPU to accelerate computing.
    \item A node of a computation graph, holding:
      \begin{itemize}
      \item the gradient w.r.t to it self (back-propagation)
      \item a reference  to its creator
      \end{itemize}
    \end{itemize}
  \end{block}


  \begin{block}{\textit{Autograd}}
    Package for building computational graphs out of Tensors, and
    automatically computing gradients
  \end{block}
  
  \begin{block}{Module}
    A neural network layer,  may store state or
    learnable  Function (i.e with parameters)
  \end{block}
\end{frame}



\begin{frame}[fragile]{An example in pytorch}
  
  \begin{minted}{python}
    import torch as th
    # The model 
    D_in=2  # input size : 2 
    D_out=1 # output size: one value 
    model = th.nn.Sequential(
       th.nn.Linear(D_in, D_out),
       th.nn.Sigmoid()    
       )
    loss_fn = th.nn.BCELoss()
    # Optimizer will update  the weights of the model. 
    lr0 = 1e-4 
    optimizer = torch.optim.SGD(model.parameters(),
                       lr=lr0) 
  \end{minted}
\end{frame}

\begin{frame}[fragile]{An example in pytorch - 2}
  \begin{minted}{python}
    for t in  range(10): 
       # Forward pass: compute predicted y by passing x.  
       y_pred = model(x)
       # Compute and print loss.  
       loss = loss_fn(y_pred, y) 
       print(t, loss.data[0])
       # Optim in two steps
       optimizer.zero_grad()
       # Backward pass: compute gradient of the loss wrt parameters 
       loss.backward()
       # Calling the step function on an Optimizer makes an update 
       optimizer.step()
  \end{minted}
\end{frame}


\begin{frame}{Three kinds of \textit{Module}}

  
  \begin{block}{Linear}
    \begin{itemize}
    \item \textit{Linear} is a \textit{Module} for a linear
      transformation.
    \item The parameters: a \textit{Tensor} $\W$
    \item Forward $\x\rightarrow \W\x$
    \end{itemize}
  \end{block}


  \begin{block}{Sigmoid}
    \begin{itemize}
    \item \textit{Sigmoid} is a \textit{Module} for a pointwise function
    \item No parameters
    \end{itemize}
  \end{block}

  \begin{block}{Sequential}
    \begin{itemize}
    \item \textit{Sequential} is a container \textit{Module} 
    \item It contains a sequence of \textit{Module}, i.e a feed-forward NNet
    \end{itemize}
  \end{block}
  \begin{center}
    Look at the \textit{torch.nn} doc for many examples
  \end{center}
\end{frame}



\begin{frame}[fragile]{From CPU to GPU}
  \begin{minted}{python}
    # This vector is stored on cpu (+any operation you do on it)
    a = torch.DoubleTensor([1., 2.])
    # The same for GPU
    a = torch.FloatTensor([1., 2.]).cuda()
    a = torch.cuda.FloatTensor([1., 2.])
    # it will be on the default device: 
    torch.cuda.current_device()
    #####
    # For the model: 
    model = model.cuda()
  \end{minted}
\end{frame}

\begin{frame}[fragile]{Define your module}
  \begin{minted}{python}
    class LogisticRegression(th.nn.Module):
        def __init__(self,D_in):
           super(LogisticRegression, self).__init__()
           self.lin = th.nn.Linear(D_in, 1)
           self.out = th.nn.Sigmoid()    
    
        def forward(self, x): 
           a = self.lin(x)
           return self.out(a)
           
    mod = LogisticRegression(D_in=2)
  \end{minted}
\end{frame}




